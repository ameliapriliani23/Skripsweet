%-------------------------------------------------------------------------------
%                      Template Naskah Skripsi
%               	Berdasarkan format JTETI FT UGM
% 						(c) @gunturdputra 2014
%-------------------------------------------------------------------------------

%Template pembuatan naskah skripsi.
\documentclass{jtetiskripsi}

%Untuk prefiks pada daftar gambar dan tabel
\usepackage[titles]{tocloft}
\renewcommand\cftfigpresnum{Gambar\  }
\renewcommand\cfttabpresnum{Tabel\   }

%Untuk hyperlink dan table of content
\usepackage{hyperref}
\newlength{\mylenf}
\settowidth{\mylenf}{\cftfigpresnum}
\setlength{\cftfignumwidth}{\dimexpr\mylenf+2em}
\setlength{\cfttabnumwidth}{\dimexpr\mylenf+2em}

%Untuk Bold Face pada Keterangan Gambar
\usepackage[labelfont=bf]{caption}

%Untuk caption dan subcaption
\usepackage{caption}
\usepackage{subcaption}

%pdf
\usepackage{pdfpages}

%table
\usepackage{graphics}

\usepackage{wrapfig}

%-----------------------------------------------------------------
%Disini awal masukan untuk data proposal skripsi
%-----------------------------------------------------------------
\titleind{Implementasi Steganografi pada Citra \emph{Digital} dengan Metode \emph{Least Significant Bit}}

\fullname{AMELIA APRILIANI}

\idnum{3145143626}

\approvaldate{02 Februari 2017}

\degree{Sarjana Ilmu Komputer}

\yearsubmit{2018}

\program{Ilmu Komputer}

\dept{Ilmu Komputer}

\firstsupervisor{Drs. Mulyono, M.Kom.}
\firstnip{1196605171994031003}

\secondsupervisor{Ratna Widyati, S.Si, M.Kom.}
\secondnip{197509252002122002}

%hypenation

\hyphenation{Al-go-rit-ma pe-san kom-pre-si di-gu-na-kan seg-men di-rep-re-sen-ta-si-kan di-kem-bang-kan di-sem-bu-nyi-kan eks-trak-si meng-gu-na-kan}

%-----------------------------------------------------------------
%Disini akhir masukan untuk data proposal skripsi
%-----------------------------------------------------------------

\begin{document}

\cover

\chapter*{\centering{\large{LEMBAR PERSETUJUAN}}}
\thispagestyle{empty} {\bf }Dengan ini saya mahasiswa Fakultas
Matematika dan Ilmu Pengetahuan Alam, Universitas Negeri Jakarta

\vskip3mm

\begin{tabular}{ll}
  Nama & : Amelia Apriliani \\
  No. Registrasi & : 3145143626 \\
  Jurusan & : Ilmu Komputer \\
  Judul & : Implementasi Steganografi pada Citra \emph{Digital} \\ & \hspace{0.2cm} dengan Metode \emph{Least Significant Bit}.
\end{tabular}

\vskip3mm

\noindent Menyatakan bahwa skripsi ini telah siap diajukan untuk seminar skripsi.
%\begin{center}
%Menyatakan bahwa skripsi ini telah siap diajukan untuk sidang skripsi.
%\end{center}



\begin{center}
\vskip3mm

Menyetujui,

\vskip3mm
\begin{spacing}{1.25}

\begin{tabular}{ccc}
  \hskip-2mm Dosen Pembimbing I & \qquad \qquad \qquad \qquad \qquad & \hskip-6mm Dosen Pembimbing II \\
   &  &  \\
   &  &  \\
   &  &  \\
   &  &  \\
  \hskip-2mm \underline{\textbf{Drs. Mulyono, M.Kom.}} &  & \hskip-6mm \underline{\textbf{Ratna Widyati, S.Si, M.Kom.}} \\
  \hskip-2mm NIP. 119660517 199403 1 003 &  & \hskip-6mm NIP. 19750925 200212 2 002	 \\
\end{tabular}
\end{spacing}
\end{center}
\vskip3mm
\begin{center}
Mengetahui, \\
Koordinator Program Studi Ilmu Komputer
\end{center}
\begin{spacing}{1.25}
{ \ }
\\
\\
{ \ }\begin{center}
\underline{\textbf{Drs. Mulyono, M.Kom.}} \\
{NIP. 119660517 199403 1 003}
\end{center}
\end{spacing} 

%-----------------------------------------------------------------
%Disini awal masukan Acknowledment
%-----------------------------------------------------------------

%-----------------------------------------------------------------
%Disini awal masukan untuk Prakata
%-----------------------------------------------------------------

%-----------------------------------------------------------------
%Disini akhir masukan untuk muka skripsi
%-----------------------------------------------------------------

\tableofcontents 
\addcontentsline{toc}{chapter}{DAFTAR ISI}
\listoffigures
\addcontentsline{toc}{chapter}{DAFTAR GAMBAR}
\listoftables
\addcontentsline{toc}{chapter}{DAFTAR TABEL}

%-----------------------------------------------------------------
%Disini awal masukan Intisari
%-----------------------------------------------------------------

%-----------------------------------------------------------------
%Disini akhir masukan Intisari
%-----------------------------------------------------------------
%-----------------------------------------------------------------
\begin{counterpage}
\end{counterpage}
%Disini awal masukan untuk Bab
%-----------------------------------------------------------------
%!TEX root = ./template-skripsi.tex
%-------------------------------------------------------------------------------
% 								BAB I
% 							LATAR BELAKANG
%-------------------------------------------------------------------------------

\chapter{LATAR BELAKANG}

\section{Latar Belakang Masalah}
Saat ini \emph{internet} sudah berkembang menjadi salah satu media yang sangat populer di berbagai dunia (Bunyamin;Adrian, 2009). Perkembangan \emph{internet} memberikan pengaruh besar terhadap kemudahan dalam berkomunikasi dan menyampaikan informasi. Komunikasi merupakan salah satu hal yang penting bagi manusia. Manusia yang merupakan makhluk sosial cenderung melakukan komunikasi setiap hari, baik secara langsung maupun melalui media elektronik. Manusia melakukan komunikasi untuk bertukar informasi.

Kemudahan dalam berkomunikasi memberikan dampak positif dan negatif. Dampak positifnya yaitu cepatnya informasi dapat tersebar, baik antar daerah maupun antar negara. Dan dampak negatifnya adalah semakin berkembangnya kejahatan dalam penggunaan informasi. Dengan berbagai teknik, banyak orang yang mencoba untuk mengakses informasi yang bukan haknya. Maka dari itu harus berkembang juga pengamanan sistem informasi.

Teknik pengamanan informasi yang ada saat ini seperti kriptografi dan steganografi. Kriptografi adalah suatu ilmu dan seni untuk menjaga kerahasiaan pesan dengan cara menyandikan ke dalam bentuk yang tidak dapat dimengerti lagi maknanya. Kriptografi telah ada dan digunakan sejak berabad-abad yang lalu dikenal dengan istilah kriptografi klasik, yang bekerja pada mode karakter alfabet  (Rakhmat;Fairuzabadi, 2010).

Steganografi adalah seni dan sains komunikasi pesan yang tak terlihat. Hal ini dilakukan dengan menyembunyikan informasi dalam informasi lain, misalnya menyembunyikan keberadaan informasi yang dikomunikasikan. Kata steganografi berasal dari kata Yunani "stegos" yang berarti "cover" dan "grafia" yang berarti "menulis" yang mendefinisikannya sebagai "tulisan tertutup"  (Mrs., Kadam, Koshti;Dunghav, 2012).

Salah satu metode steganografi adalah \emph{Least Significant Bit} (LSB). Algoritma LSB, menggantikan bit paling signifikan pada \emph{file cover} sesuai dengan bit pesan. Teknik ini adalah teknik yang paling populer digunakan dalam steganografi untuk menyembunyikan pesan. Teknik ini biasanya efektif, karena substitusi LSB tidak menyebabkan degradasi kualitas yang signifikan  (Joshi;Yadav, 2015).

\section{Batasan Masalah}
Batasan masalah dalam tugas akhir ini mencakup:
\begin{itemize}
	\item \emph{Software} yang digunakan adalah Matlab R2013b.
	\item  Peneliti akan langsung menggunakan \emph{smartphone} berbasis Android dalam proses \emph{debugging} dan \emph{testing}. Versi android yang digunakan adalah Lollipop (Android 5.0) dengan API 21.
	\item Format \emph{file}  citra \emph{digital} yang dapat digunakan untuk menyimpan pesan adalah berformat *.bmp.
	\item Format \emph{file}  citra \emph{digital} yang dihasilkan dari program steganografi ini adalah berformat *.bmp.
	\item Pesan yang dapat disimpan hanya berformat *.txt.
	\item Metode yang digunakan adalah \emph{Least Signifiant Bit}.
\end{itemize}

\section{Rumusan Masalah}
Rumusan masalah berdasarkan latar belakang di atas adalah:
\begin{enumerate}
	\item Bagaimana cara menyembunyikan teks dalam proses steganografi dengan menggunakan metode \emph{Least Significant Bit}?
	\item Bagaimana perubahan dalam \emph{file} citra hasil keluaran sebelum dan sesudah disisipkan pesan teks?
\end{enumerate}


\section{Tujuan Penelitian}
Tujuan dari penelitian ini adalah: 
\begin{enumerate}
	\item Memberikan informasi bagaimana teknik steganografi dapat diterapkan untuk menyembunyikan teks dalam \emph{file} citra \emph{digital} dengan menggunakan metode \emph{Least Significant Bit}. 
	\item Mengetahui perubahan yang terjadi dari hasil keluaran \emph{file} citra \emph{digital}.
\end{enumerate}

\section{Manfaat Penelitian}
Penelitian ini diharapkan memberikan manfaat sebagai berikut:
	\begin{enumerate}
		\item Bagi Penulis, diharapkan dapat menambah pengetahuan dan pemahaman tentang steganografi.
		\item Bagi Program Studi Ilmu Komputer, Penulisan penelitian ini memberikan gambaran bagi seluruh mahasiswa khususnya bagi mahasiswa program studi Ilmu Komputer Universitas Negeri Jakarta tentang bagaimana teknik stegaografi dapat menyembunyikan pesan dalam \emph{file} citra \emph{digital}.  	
	\end{enumerate}

\section{Jenis Penelitian}
Jenis Penelitian yang dijalani oleh Peneliti berjenis ... Jenis penelitian ini mengarahkan penulis kepada ...		
% Baris ini digunakan untuk membantu dalam melakukan sitasi
% Karena diapit dengan comment, maka baris ini akan diabaikan
% oleh compiler LaTeX.
\begin{comment}
\bibliography{daftar-pustaka}
\end{comment}


%!TEX root = ./template-skripsi.tex
%-------------------------------------------------------------------------------
%                            BAB II
%               KAJIAN TEORI
%-------------------------------------------------------------------------------

\chapter{KAJIAN TEORI}                

\section{Steganografi}
	\subsection{Pengertian Steganografi}
	Menurut \textbf{Ir. Rinaldi Munir, M.T.} dalam Diktat Kuliah Kriptografi dengan judul Steganografi dan \emph{Watermaking}:
	
	"Steganografi (\emph{steganography}) adalah ilmu dan seni menyembunyikan pesan rahasia (\emph{hiding message}) sedemikian sehingga keberadaan (eksistensi) pesan tidak terdeteksi oleh indera manusia." \cite{munir}
	
	Menurut \textbf{Gary C. Kessler} dalam jurnalnya \emph{Steganography Hiding Data Within Data}:
	
	"Steganografi adalah ilmu menyembunyikan informasi. Tujuan steganografi adalah untuk menyembunyikan data dari pihak ketiga." \cite{kessler}
	
	\begin{figure}[H]
		\centering
		\includegraphics[width=1\textwidth]{gambar/diagram_steganografi}
		\caption{Diagram penyisipan dan ekstraksi pada pesan}
		\label{diagram_steganografi}
	\end{figure} 
	
	Istilah di dalam steganografi:
	\begin{enumerate}
		\item \emph{Covertext} merupakan media atau tempat pesan yang digunakan untuk menyembunyikan \emph{hiddentext}. \emph{Covertext} bisa berupa teks, gambar, audio, video, dll.
		\item \emph{Hiddentext}	atau biasa disebut \emph{embedded message} merupakan pesan atau informasi yang ingin disembunyikan. Contohnya bisa berupa teks, gambar, audio, video, dll.
		\item \emph{Stegotext} merupakan pesan yang sudah berisi \emph{embedded message}.
		\item \emph{Encoding} yaitu penyisipan pesan ke dalam media \emph{covertext}.
		\item \emph{Decoding} yaitu ekstraksi pesan dari \emph{stegotext}.
	\end{enumerate}
	
	Menurut \textbf{Munir}, ada kriteria yang harus diperhatikan dalam penyembunyian pesan, yaitu meliputi \emph{Imperceptible}, \emph{Fidelity}, \emph{Recovery} dan \emph{Capacity}.
	\begin{enumerate}
		\item \emph{Imperceptible}\\ 
		Keberadaan pesan rahasia tidak dapat dipersepsi secara visual atau secara audio. Jika \emph{covertext} berupa \emph{file} citra, maka \emph{stegotext} yang dihasilkan harus sukar dibedakan oleh kasat mata dengan \emph{covertext}-nya. Dan jika \emph{covertext} berupa \emph{file} audio, maka telinga tidak dapat mendeteksi perubahan yang ada pada audio \emph{stegotext}-nya. 
		\item \emph{Fidelity}\\
		Kualitas \emph{file} citra penampung tidak jauh berubah. Setelah penambahan pesan rahasia, citra hasil steganografi masih terlihat dengan baik. Pengamat tidak mengetahui kalau di dalam citra tersebut terdapat pesan rahasia.
		\item \emph{Recovery}\\
		Pesan yang disembunyikan harus dapat diekstrak kembali. Karena tujuan steganografi adalah menyembunyikan pesan atau informasi, maka jika informasi itu dibutuhkan harus dapat diambil kembali untuk dapat digunakan.
		\item \emph{Capacity}\\
		Ukuran pesan yang akan disembunyikan sedapat mungkin besar. Agar dapat memaksimalkan manfaat dari steganografi itu sendiri. \cite{munir}
	\end{enumerate}
	
	\subsection{Sejarah Steganografi}
	Seperti kriptografi, penggunaan steganografi sebetulnya telah digunakan berabad-abad yang lalu bahkan sebelum istilah steganografi itu sendiri muncul. Periode sejarah steganografi dapat dibagi menjadi:
	\begin{enumerate}
		\item Steganografi Kuno (\emph{Ancient Steganography})
			\begin{enumerate}
				\item Steganografi dengan media kepala budak
				
				Ditulis oleh \textbf{Herodatus} (485–525 BC), sejarawan Yunani pada tahun 440 BC di dalam buku: \emph{Histories of Herodatus}). Kisah perang antara kerajaan Persia dan rakyat Yunani. \textbf{Herodatus} menceritakan cara \textbf{Histaiaeus} mengirim pesan kepada \textbf{Aristagoras of Miletus} untuk melawan Persia. 
				
				Caranya adalah dengan dipilih beberapa budak. Kemudian kepala budak tersebut digunduli dan ditulis pesan dengan cara ditato. Setelah pesan dituliskan, budak harus menunggu hingga rambutnya tumbuh kembali. Setelah rambut pada kepala budak tersebut tumbuh, budak dikirim ke tempat penerima. Di sana kepala budak digunduli agar pesan dapat dibaca.
					\begin{figure}[H]
						\centering
						\includegraphics[width=1\textwidth]{gambar/steganografi_kepalabudak}
						\caption{Steganografi dengan media kepala budak}
						\label{steganografi_kepalabudak}
					\end{figure}
				
				\item Penggunaan tablet \emph{wax}
				
				Orang-orang Yunani kuno menulis pesan rahasia di atas kayu yang kemudian ditutup dengan lilin (\emph{wax}). Di dalam bukunya, \textbf{Heradatus} menceritakan \textbf{Demaratus} mengirim peringatan tentang serangan yang akan datang ke Yunani dengan menulis langsung pada tablet kayu yang kemudian dilapisi lilin dari lebah.
					\begin{figure}[H]
						\centering
						\includegraphics[width=1\textwidth]{gambar/tablet_wax}
						\caption{Tablet \emph{wax}}
						\label{tablet_wax}
					\end{figure}
					
				\item Penggunaan tinta tak-tampak (\emph{invisible ink})
				
				\textbf{Pliny the Elder} menjelaskan penggunaan tinta dari getah tanaman \emph{thithymallus}. Jika dituliskan pada kertas maka tulisan dengan tinta tersebut tidak kelihatan, tetapi bila kertas dipanaskan berubah menjadi gelap/coklat.
				
				\item Penggunaan kain sutra dan lilin
				
				Orang Cina kuno menulis catatan pada potongan-potongan kecil sutra yang kemudian digumpalkan menjadi bola kecil dan dilapisi lilin. Selanjutnya bola kecil tersebut ditelan oleh si pembawa pesan. Pesan dibaca setelah bola kecil dikeluarkan dari perut si pembawa pesan.			
			\end{enumerate}
		\item Steganografi Zaman Renaisans (\emph{Renaissance Steganography})
		
		Tahun 1499, \textbf{Johannes Trithemius} menulis buku \emph{Steganographia}, yang menceritakan tentang metode steganografi berbasis karakter. Selanjutnya tahun 1518 dia menulis buku tentang steganografi dan kriptografi berjudul \emph{Polygraphiae}.
		
		\textbf{Giovanni Battista Porta} menggambarkan cara menyembunyikan pesan di dalam telur rebus. Caranya, pesan ditulis pada kulit telur yang dibuat dari tinta khusus yang dibuat dengan satu ons tawas dan setengah liter cuka. Prinsipnya penyembunyiannya adalah tinta tersebut akan menembus kulit telur yang berpori, tanpa meninggalkan jejak yang terlihat. Tulisan dari tinta akan membekas pada permukaan isi telur yang telah mengeras (karena sudah direbus sebelumnya). Pesan dibaca dengan membuang kulit telur.
		
		\item Steganografi Zaman Perang Dunia (\emph{World War Steganography})
		
		\begin{figure}[H]
			\centering
			\includegraphics[width=1\textwidth]{gambar/steganografi_perangdunia}
			\caption{Steganografi zaman perang dunia}
			\label{steganografi_perangdunia}
		\end{figure}
	
		Selama terjadinya Perang Dunia ke-2, tinta yang tidak tampak (\emph{invisible ink}) telah digunakan untuk menulis informasi pada lembaran kertas sehingga saat kertas tersebut jatuh di tangan pihak lain hanya akan tampak seperti lembaran kertas kosong biasa. Cairan seperti air kencing (\emph{urine}), susu, vinegar, dan jus buah digunakan sebagai media penulisan sebab bila salah satu elemen tersebut dipanaskan, tulisan akan menggelap dan tampak melalui mata manusia. \cite{munir}
		
		\item Steganografi \emph{Digital}
		
		Sejalan dengan perkembangan maka konsep awal steganografi diimplementasikan pula dalam dunia komputer, yang kemudian dikenal dengan istilah steganografi \emph{digital}. Dalam hal ini, steganografi \emph{digital} memiliki dua properti dasar yaitu media penampung (\emph{cover data} atau \emph{data carrier}) dan data \emph{digital} yang akan disisipkan (\emph{secret data}), dimana media penampung dan data \emph{digital} yang akan disisipkan dapat berupa \emph{file} multimedia (teks/dokumen, citra, audio maupun video). Terdapat dua tahapan umum dalam steganografi \emph{digital}, yaitu proses \emph{embedding} atau \emph{encoding} (penyisipan) dan proses \emph{extracting} atau \emph{decoding} (pemekaran atau pengungkapan kembali (\emph{reveal})). Hasil yang didapat setelah proses \emph{embedding} atau \emph{encoding} disebut \emph{stego object} (apabila media penampung hanya berupa data citra maka disebut \emph{stego image}). \cite{prayudi}
	\end{enumerate}

	\subsection{Metode Steganografi}
	Berdasarkan ranah operasinya, metode-metode steganografi dapat dibagi menjadi dua kelompok:
	\begin{enumerate}
		\item \emph{Spatial (time) domain methods}\\
		Memodifikasi langsung nilai byte dari \emph{cover-object} (nilai \emph{byte} dapat merepresentasikan intensitas/warna \emph{pixel} atau amplitudo). Contoh: Metode modifikasi LSB
		
		\item \emph{Tranform domain methods}\\
		Memodifikasi hasil transformasi sinyal dalam ranah transform (hasil trnasformasi dari ranah spasial ke ranah lain (misalnya ranah frekuensi). Contoh: Metode \emph{Spread Spectrum} \cite{munir}
		
	\end{enumerate}

	Ada empat jenis metode steganografi:
	\begin{enumerate}
		\item \emph{Least Significant Bit Insertion} (LSB)\\
		Metode yang digunakan untuk menyembunyikan pesan pada media \emph{digital} tersebut berbeda-beda. Contohnya, pada berkas \emph{image} pesan dapat disembunyikan dengan menggunakan cara menyisipkannya pada bit rendah atau bit yang paling kanan (LSB) pada data \emph{pixel} yang menyusun \emph{file} tersebut. Pada berkas \emph{bitmap} 24 bit, setiap \emph{pixel} (titik) pada gambar tersebut terdiri dari susunan tiga warna \emph{Red}, \emph{Green} dan \emph{Blue} (RGB) yang masing-masing disusun oleh bilangan 8 bit (\emph{byte}) dari 0 sampai 255 atau dengan format biner 00000000 sampai 11111111. Dengan demikian, pada setiap \emph{pixel} berkas \emph{bitmap} 24 bit kita dapat menyisipkan 3 bit data. 
		\item \emph{Algorithms and Transformation}\\
		\emph{Algoritma compression} adalah metode steganografi dengan menyembunyikan data dalam fungsi matematika. Dua fungsi tersebut adalah \emph{Discrete Cosine Transformation} (DCT) dan \emph{Wavelet Transformation}. Fungsi DCT dan \emph{Wavelet} yaitu mentransformasi data dari satu tempat (\emph{domain}) ke tempat (\emph{domain}) yang lain. Fungsi DCT yaitu mentransformasi data dari tempat \emph{spatial} (\emph{spatial domain}) ke tempat frekuensi (\emph{frequency domain}).
		\item \emph{Redundant Pattern Encoding}
		\emph{Redundant Pattern Encoding} adalah menggambar pesan kecil pada kebanyakan gambar. Keuntungan dari metode ini adalah dapat bertahan dari \emph{cropping} (kegagalan). Kerugiannya yaitu tidak dapat menggambar pesan yang lebih besar.
		\item \emph{Spread Spectrum Method}
		\emph{Spread Spectrum} steganografi terpencar-pencar sebagai pesan yang diacak (\emph{encrypted}) melalui gambar (tidak seperti dalam LSB). Untuk membaca suatu pesan, penerima memerlukan algoritma yaitu \emph{crypto-key} dan \emph{stego-key}. Metode ini juga masih mudah diserang yaitu penghancuran atau pengrusakan dari kompresi dan proses \emph{image} (gambar)	
	\end{enumerate}

\section{Perbedaan Steganografi dan Kriptografi}
\section{LSB (\emph{Least Significant Bit})}
\section{ASCII}
\section{\emph{Citra Digital}}
	\subsection{Pengertian Citra \emph{Digital}}
	\subsection{Pengolahan Citra \emph{Image Processing}}
	\subsection{Perbandingan \emph{File} Gambar BMP (\emph{Bitmap}) dengan JPG, GIF, atau PNG}
\section{MATLAB}	



% Baris ini digunakan untuk membantu dalam melakukan sitasi
% Karena diapit dengan comment, maka baris ini akan diabaikan
% oleh compiler LaTeX.
\begin{comment}
bibliography{daftar-pustaka}
\end{comment}


%!TEX root = ./template-skripsi.tex
%-------------------------------------------------------------------------------
%                            BAB III
%               		PEMBAHASAN
%-------------------------------------------------------------------------------

\chapter{PEMBAHASAN}

\section{Rancangan Algoritma}
Pada pembuatan steganografi dengan metode LSB ini menggunakan bantuan MATLAB. Dan citra \emph{digital} yang digunakan adalah dengan format BMP, alasan dipilihnya citra dengan format BMP adalah karena kesesuaiannya dengan metode LSB. Dan pesan yang disisipkan adalah pesan dengan format \emph{text}

	\subsection{Proses Penyisipan (\emph{Encoding}) pesan ke Citra \emph{Digital}}
	Proses penyisipan pesan ke citra yaitu:
	\begin{enumerate}
		\item Siapkan \emph{Cover Image} yang akan disisipkan \emph{Hiddentext}
		\item Masukkan \emph{Hiddentext} yang akan disisipkan
		\item \emph{Cover Image} dikonversi setiap nilai \emph{pixel}-nya ke biner, sedangkan \emph{Hiddentext} dikonversi dari huruf ke ASCII dan dikonversi lagi menjadi biner
		\item Proses \emph{Encoding} dilakukan dengan LSB
		\item Selanjutnya proses konversi dari biner ke \emph{pixel} yang menghasilkan \emph{Stego Image}
		\item Hasil dari citra yang telah disisipkan pesan atau \emph{Stego Image} disimpan
	\end{enumerate}
	
	\begin{figure}[H]
		\centering
		\includegraphics[width=1\textwidth]{gambar/penyisipan2}
		\caption{\emph{Flowchart} Penyisipan Pesan Rahasia}
		\label{flowchart_penyisipan}
	\end{figure}
	
	\subsection{Proses Ekstraksi (\emph{Decoding}) pesan dari Citra \emph{Digital}}
	Proses ekstraksi pesan dari citra yaitu:
	\begin{enumerate}
		\item Siapkan \emph{Stego Image} yang telah disisipkan \emph{Hiddentext}
		\item Konversi \emph{pixel} ke biner
		\item Proses \emph{Decoding} dilakukan dengan LSB
		\item Konversi biner ke \emph{pixel} untuk mendapatkan \emph{Cover Image}, dan konversi biner ke ASCII kemudian ke huruf untuk mendapatkan pesan rahasia atau \emph{Hiddentext}
		\item \emph{Hiddentext} didapatkan
	\end{enumerate}
	
	\begin{figure}[H]
		\centering
		\includegraphics[width=1\textwidth]{gambar/ekstraksi2}
		\caption{\emph{Flowchart} Ekstraksi Pesan Rahasia}
		\label{flowchart_ekstraksi}
	\end{figure}
	
\section{Desain Antar Muka Program}
\section{Perbedaan Citra Sebelum dan Sesudah disisipkan Pesan}


%%!TEX root = ./template-skripsi.tex
%-------------------------------------------------------------------------------
%                            	BAB IV
%               		KESIMPULAN DAN SARAN
%-------------------------------------------------------------------------------

\chapter{KESIMPULAN DAN SARAN}

\section{Kesimpulan}
Berdasarkan hasil implementasi dan pengujian program ini, didapat kesimpulan sebagai berikut:

\begin{enumerate}
	\item Satu
	
	\item Dua
	
	\item Tiga

\end{enumerate}


\section{Saran}
Adapun saran-saran penulis untuk penelitian selanjutnya adalah:
\begin{enumerate}
	\item Satu
	
	\item Dua
	
	\item Tiga 
	
\end{enumerate}


% Baris ini digunakan untuk membantu dalam melakukan sitasi
% Karena diapit dengan comment, maka baris ini akan diabaikan
% oleh compiler LaTeX.
\begin{comment}
\bibliography{daftar-pustaka}
\end{comment}

%\include{bab5}
%-----------------------------------------------------------------
%Disini akhir masukan Bab
%-----------------------------------------------------------------


%-----------------------------------------------------------------
% Disini awal masukan untuk Daftar Pustaka
% - Daftar pustaka diambil dari file .bib yang ada pada folder ini
%   juga.
% - Untuk memudahkan dalam memanajemen dan menggenerate file .bib
%   gunakan reference manager seperti Mendeley, Zotero, EndNote,
%   dll.
%-----------------------------------------------------------------
%\bibliography{IEEEabrv,daftar-pustaka}
\begin{thebibliography}{99}
	\bibitem{adiria}Adiria. (2010). "Analisis dan Perancangan Aplikasi Steganografi pada Citra Digital dengan Menggunakan Metode LSB (Least Significant Bit)". Skripsi Sarjana pada Universitas Islam Negeri Jakarta
	
	\bibitem{arymurthy}Arymurthy, A. M., dan Setiawan, S. (1992). "Pengantar Pengolahan Citra. Jakarta: PT Elex Media Komputindo".
	
	\bibitem{ascii} ASCII Table.  2010. "ASCII Table and Description".  ASCII Table [Online]. Tersedia: \url{https://www.asciitable.com}. [17 April 2018].
	
	\bibitem{bunyamin}Bunyamin, H., dan Adrian. (2009). "Aplikasi Steganography pada File dengan Menggunakan Teknik Low Bit Encoding dan Least Significant Bit". Jurnal Informatika UKM, Vol. 5, No. 2, pp. 107–117.
	
	\bibitem{elgabar}Elgabar, Eltyeb E. A bed. (2013). "Comparison of LSB Steganography in BMP and JPEG Images". International Journal of Soft Computing and Engineering (IJSCE), ISSN: 2231-2307, Vol.3, Issue-5.
	
	\bibitem{elgabar2}Elgabar, Eltyeb E. A bed dan Mohammed, Fakhreldeen A. (2013). "JPEG versus GIF Images in forms of LSB Steganography".  International Journal of Computer Science and Network (IJCSN), Vol. 2, Issue 6.
	
	\bibitem{gautam}Gautama, Prakriti dan Sharma, Deepak. (2015). "A Survey on Digital Image Steganography Techniques". International Journal of Electronics, Electrical and Computational System (IJEECS), ISSN 2348-117X, Vol. 4, Issue 11. 
	
	\bibitem{hermawati}Hermawati, F. A. (2013). "Pengolahan Citra Digital". Yogyakarta: ANDI.
	
	\bibitem{irfan}Irfan. (2013). "Penyembunyian Informasi (steganography) Gambar Menggunakan Metode LSB (Least Significant Bit)". Rekayasa Teknologi Vol. 5, No. 1.
	
	\bibitem{joshi}Joshi, K., dan Yadav, R. (2015). "A New LSB-S Image Steganography Method Blend with Cryptography for Secret Communication". Third International Conference on Image Infomation Processing.
	
	\bibitem{kessler}Kessler, G. C. (2001). "Steganography Hiding Data Within Data".
	
	\bibitem{kadam}M. K., Kadam, K., Koshti, A., dan Dunghav, P. (2012). "Steganography Using Least Signicant Bit Algorithm". International Journal of Engineering Research and Applications (IJERA), Vol. 2, Issue 3, pp. 338-341.
	
	\bibitem{munir04}Munir, R. (2004). "Pengolahan Citra Digital". Bandung: Informatika.
	
	\bibitem{munir}Munir, R. (2006). "Kriptografi". Bandung: Informatika.
	
	\bibitem{pakereng}Pakereng, M.A Ineke, Beeh, Yos Richard, dan Endrawan, Sonny. (2010). "Perbandingan Steganografi Metode Spread Spectrum dan Least Significant Bit (LSB) Antara Waktu Proses dan Ukuran File Gambar". JURNAL INFORMATIKA, VOL. 6 NO. 1.
	
	\bibitem{pavani}M., Pavani, S., Naganjaneyulu, dan C., Nagaraju. (2013). "A Survey on LSB Based Steganography Methods". International Journal Of Engineering And Computer Science, Vol. 2, Issue pp. 2464-2467.
	
	\bibitem{prasetyo}Prasetyo, F. P. (2010). "Steganografi Menggunakan Metode LSB denga Software Matlab". Skripsi Sarjana pada Universitas Islam Negeri Jakarta
	
	\bibitem{prayudi}Prayudi, Y., dan Kuncoro, P. S. (2005). "Implementasi Steganografi Menggunakan Teknik Adaptive Minimum Error Least Significant Bit Replacement (AMELSBR)". Seminar Nasional Aplikasi Teknologi Informasi.
	
	\bibitem{rakhmat}Rakhmat, B., dan Fairuzabadi, M. (2010). "Steganografi Menggunakan Metode Least Significant Bit dengan Kombinasi Algoritma Kriptografi Vigenere dan RC4". Jurnal Dinamika Informatika, Volume 5, Nomor 2.
	
	\bibitem{setiana}Setiana, dan Mahmudy, W. F. (2006). "Steganografi Pada File Citra Bitmap 24 Bit Untuk Pengamanan Data Menggunakan Metode Least Significant Bit (LSB) Insertion". Kursor, vol. 2, no. 2, pp. 38-44.
	
	\bibitem{wikipedia1}Wikipedia. (n.d.). Retrieved from https://id.wikipedia.org/wiki/Steganografi
		
		
	
\end{thebibliography}
\addcontentsline{toc}{chapter}{DAFTAR PUSTAKA}
%-----------------------------------------------------------------
%Disini akhir masukan Daftar Pustaka
%-----------------------------------------------------------------

%\addcontentsline{toc}{chapter}{LAMPIRAN}
\appendix 
\chapter{\emph{Source Code}}
	\begin{verbatim}
		char_max = (row -1)*(column);
		char_max = round((char_max*3)/8);
		
		%Cek Kondisi Lokasi
		lokasi = get(handles.kolom_lokasi, 'String')
		if isempty(lokasi) %cek kondisi axes
		msgbox('Gambar belum dimasukkan','Peringatan','warn');
		return;    
		end		
	
		%Cek Kondisi Kolom Hiddentext
		hiddentext = get(handles.edit_pesan,'String');
		if isempty(hiddentext)
			msgbox('Pesan belum dimasukkan','Peringatan','warn');
		return;    
		end
	
		%Menghitung Panjang Pesan
		row_max = row;
		column_max = column;
		hiddentext_length = length(hiddentext) %masih dalam hitungan desimal
		if hiddentext_length < char_max 
		hiddentext_biner = strcat(reshape(dec2bin(double(hiddentext),8).',1,[]), '00000000')
		hiddentext_save = hiddentext_biner;
		hiddentext_asli = char(bin2dec(reshape(hiddentext_biner,8,[]).')).'
		
		else
		msgbox('Maaf,pesan terlalu panjang','peringatan','warn');
		return;
		end
	
		%Encoding
		hiddentext_length = hiddentext_length*8;
		for i = 1:row_max-1
		for j = 1:column_max    
		if hiddentext_length ~= 0
		image_biner_red = dec2bin(image_red(i,j),8); %11100100 contoh biner
		image_biner_red(1,8) = hiddentext_biner(1,1); %bit terakhir diganti dengan bit  pesan
		image_red(i,j) = bin2dec(image_biner_red); 
		
		hiddentext_biner(1:1) = [];  %menghapus 1 bit pertama pesan
		hiddentext_length = length(hiddentext_biner); %menghitung panjang bit pesan            
		end
		
		if hiddentext_length ~= 0
		image_biner_green = dec2bin(image_green(i,j),8);
		image_biner_green(1,8) = hiddentext_biner(1,1);
		image_green(i,j) = bin2dec(image_biner_green);
		
		hiddentext_biner(1:1) = []; 
		hiddentext_length = length(hiddentext_biner); 
		end
		
		if hiddentext_length ~= 0
		image_biner_blue = dec2bin(image_blue(i,j),8);
		image_biner_blue(1,8) = hiddentext_biner(1,1); 
		image_blue(i,j) = bin2dec(image_biner_blue); 
		
		hiddentext_biner(1:1) = []; 
		hiddentext_length = length(hiddentext_biner); 
		end
		end
		end
		
		stego_image(:,:,1) = uint8(image_red);
		stego_image(:,:,2) = uint8(image_green);
		stego_image(:,:,3) = uint8(image_blue);
		
		[nama_file, direktori] = uiputfile('*.bmp','Simpan Stego Image');
		if direktori == 0
		return;
		end
		nama = fullfile(direktori, nama_file);
		imwrite(stego_image, nama, 'bmp');
		msgbox('Stego Image telah berhasil dibuat','pemberitahuan');	
	
		%Decoding
		hiddentext = '';
		var_null = 1;
		for i = 1:row_max-1
		for j = 1:column_max
		biner_length = length(hiddentext);
		if biner_length < hiddentext_length && var_null<=8
		image_biner_red = dec2bin(image_red(i,j),8); %11100101 binernya 
		hiddentext_red = image_biner_red(1,8); %diambil '1' dari bit yang terakhir
		hiddentext = strcat(hiddentext, hiddentext_red); %digabungin sama si hiddentext yg awalnya kosong
		if hiddentext_red == 0
		var_null = var_null + 1;
		else
		var_null = 1;
		end   
		else
		hiddentext_asli = char(bin2dec(reshape(hiddentext,8,[]).')).';
		set(handles.edit_pesan,'String',hiddentext_asli);
		return;
		end
		
		biner_length = length(hiddentext);
		if biner_length < hiddentext_length && var_null<=8
		image_biner_green = dec2bin(image_green(i,j),8);
		hiddentext_green = image_biner_green(1,8);
		hiddentext = strcat(hiddentext, hiddentext_green);
		if hiddentext_green == 0
		var_null = var_null+1;
		else
		var_null = 1;
		end   
		else
		hiddentext_asli = char(bin2dec(reshape(hiddentext,8,[]).')).';
		set(handles.edit_pesan,'String',hiddentext_asli);
		return;
		end
		
		biner_length = length(hiddentext);
		if biner_length < hiddentext_length && var_null<=8
		image_biner_blue = dec2bin(image_blue(i,j),8);
		hiddentext_blue = image_biner_blue(1,8);
		hiddentext = strcat(hiddentext, hiddentext_blue);
		if hiddentext_blue == 0
		var_null = var_null+1;
		else
		var_null = 1;
		end   
		else
		hiddentext_asli = char(bin2dec(reshape(hiddentext,8,[]).')).';
		set(handles.edit_pesan,'String',hiddentext_asli);
		return;
		end
		
		end
		end
	\end{verbatim}


%\include{cv}

\end{document}