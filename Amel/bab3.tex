%!TEX root = ./template-skripsi.tex
%-------------------------------------------------------------------------------
%                            BAB III
%               		PEMBAHASAN
%-------------------------------------------------------------------------------

\chapter{METODOLOGI PENELITIAN}

Metodologi penelitian adalah cara atau teknik yang disusun secara teratur yang digunakan oleh seorang peneliti untuk mengumpulkan data/informasi dalam melakukan penelitian yang disesuaikan dengan subjek/objek yang diteliti. Selain metode pengumpulan data, juga dibutuhkan metode dalam pengembangan sistem. 
\section{Metode Pengumpulan Data}
	\subsection{Studi Pustaka}
	Penulis mendapatkan informasi yang berkaitan dengan steganografi melalui buku referensi dan juga dalam bentuk \emph{e-book}. Penulis juga mencari informasi melalui berbagai situs di internet yang sesuai dengan topik.	
	\subsection{Studi Literatur}
	Penulis mencoba mencari perbandingan dengan studi sejenis dari beberapa karya ilmiah lokal maupun internasional, seperti jurnal dan skripsi.

\section{Metode Pengembangan Sistem}
Dalam menyusun tugas akhir ini Penulis menggunakan metode \emph{Prototype}. Model \emph{Prototype} dimulai dengan pengumpulan kebutuhan. Pendekatan \emph{prototyping model} digunakan jika pemakai hanya mendefenisikan objektif umum dari perangkat lunak tanpa merinci kebutuhan \emph{input}, pemrosesan dan \emph{output}-nya, sementara pengembang tidak begitu yakin akan efisiensi algoritma, adaptasi sistem operasi, atau bentuk antarmuka manusia-mesin yang harus diambil. 
	\subsection{Analisis Kebutuhan}
	Pada tahapan ini Penulis melakukan pengumpulan fakta-fakta kebutuhan yang mendukung perancangan sistem dengan mengadakan konsultasi dengan dosen pembimbing maupun dosen yang berkemampuan dalam bidang ini .
	\subsection{Perancangan dan Implementasi}
	Perancangan dan implementasi steganografi pada citra \emph{digital} dengan metode \emph{Least Significant Bit} dibantu dengan menggunakan alat bantu aplikasi Matlab R2016b.
	\subsection{Pengujian}
	Pada tahap ini sistem yang sudah dirancang oleh perancang kemudian diuji oleh perancang, dosen pembimbing maupun dosen yang berkemampuan dalam bidang ini.