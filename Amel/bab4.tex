%!TEX root = ./template-skripsi.tex
%-------------------------------------------------------------------------------
%                            	BAB IV
%               		KESIMPULAN DAN SARAN
%-------------------------------------------------------------------------------

\chapter{KESIMPULAN DAN SARAN}

\section{Kesimpulan}
Berdasarkan hasil implementasi dan pengujian program ini, didapat kesimpulan sebagai berikut:

\begin{enumerate}
	\item Format \emph{file} citra yang bisa digunakan adalah format *.bmp 24 bit dan 32 bit karena mengandung RGB (\emph{Red, Green, Blue}), sedangkan pada \emph{file} citra 8 bit tidak dapat diproses karena tidak mengandung RGB.
	
	\item Pesan dalam format teks berhasil untuk disisipkan ke dalam \emph{file} citra dengan syarat tidak melebihi kapasitas maksimal dari \emph{file} citra tersebut.
	
	\item \emph{File} citra yang digunakan baik sebelum dan sesudah diproses tidak berubah. Perubahan yang terjadi tidak dapat dilihat secara visual.
	
	\item Pesan dalam \emph{Stego Image} dapat ditampilkan kembali dan sesuai dengan pesan awal. Kecuali ketika \emph{Stego Image} di-\emph{crop} maka pesan asli yang terdapat di \emph{file} citra tersebut tidak dapat ditampilkan kembali.

\end{enumerate}


\section{Saran}
Adapun saran-saran penulis untuk penelitian selanjutnya adalah:
\begin{enumerate}
	\item Program ini hanya dapat menyisipkan pesan berupa teks, diharapkan untuk kedepannya dikembangkan sehingga dapat menyisipkan \emph{file}, gambar atau audio.
	
	\item Media yang digunakan berupa \emph{file} citra, diharapkan dapat menggunakan media lain seperti audio.
	
	\item Program steganografi ini masih dikembangkan dengan MATLAB untuk perangkat komputer, diharapkan dapat dikembangkan lebih lanjut untuk perangkat \emph{mobile}. 
	
\end{enumerate}


% Baris ini digunakan untuk membantu dalam melakukan sitasi
% Karena diapit dengan comment, maka baris ini akan diabaikan
% oleh compiler LaTeX.
\begin{comment}
\bibliography{daftar-pustaka}
\end{comment}