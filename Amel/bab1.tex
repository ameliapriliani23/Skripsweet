%!TEX root = ./template-skripsi.tex
%-------------------------------------------------------------------------------
% 								BAB I
% 							LATAR BELAKANG
%-------------------------------------------------------------------------------

\chapter{LATAR BELAKANG}

\section{Latar Belakang}
Saat ini \emph{internet} sudah berkembang menjadi salah satu media yang sangat populer di berbagai dunia \cite{bunyamin}. Perkembangan \emph{internet} memberikan pengaruh besar terhadap kemudahan dalam berkomunikasi dan menyampaikan informasi. Komunikasi merupakan salah satu hal yang penting bagi manusia. Manusia yang merupakan makhluk sosial cenderung melakukan komunikasi setiap hari, baik secara langsung maupun melalui media elektronik. Manusia melakukan komunikasi untuk bertukar informasi.

Kemudahan dalam berkomunikasi memberikan dampak positif dan negatif. Dampak positifnya yaitu cepatnya informasi dapat tersebar, baik antar daerah maupun antar negara, sedangkan dampak negatifnya adalah semakin berkembangnya kejahatan dalam penggunaan informasi. Dengan berbagai teknik, banyak orang yang mencoba untuk mengakses informasi yang bukan haknya. Oleh karena itu harus berkembang juga pengamanan sistem informasi.

Teknik pengamanan informasi yang ada saat ini seperti kriptografi dan steganografi. Kriptografi adalah ilmu dan seni untuk menjaga kerahasiaan pesan dengan cara menyandikan pesan  ke dalam bentuk yang tidak dapat dimengerti lagi maknanya. Kriptografi telah ada dan digunakan sejak berabad-abad yang lalu dikenal dengan istilah kriptografi klasik, yang bekerja pada mode karakter alfabet  \cite{rakhmat}.

Steganografi adalah seni dan sains komunikasi pesan yang tidak terlihat. Hal ini dilakukan dengan menyembunyikan informasi dalam informasi lain, misalnya menyembunyikan keberadaan informasi yang dikomunikasikan. Kata steganografi berasal dari kata Yunani "stegos" yang berarti "cover" dan "grafia" yang berarti "menulis" yang mendefinisikannya sebagai "tulisan tertutup" \cite{kadam}. Steganografi merupakan teknik keamanan yang kuat, terutama ketika dikombinasikan dengan citra \emph{digital} \cite{gautam}.

Salah satu metode steganografi adalah \emph{Least Significant Bit} (LSB). Algoritma LSB, menggantikan bit paling signifikan pada \emph{file cover} sesuai dengan bit pesan. Teknik ini adalah teknik yang paling populer digunakan dalam steganografi untuk menyembunyikan pesan. Teknik ini biasanya efektif, karena substitusi LSB tidak menyebabkan degradasi kualitas yang signifikan \cite{joshi}. Kelebihan dari LSB adalah algoritmanya sederhana dan proses didalamnya \emph{relative} cepat \cite{pavani}.

Pengimplementasian metode Least Significant Bit pada steganografi sudah pernah dilakukan penelitian oleh Fahri Perdana Prasetyo dengan format file *.TIFF menggunakan bahasa pemrograman MATLAB \cite{prasetyo}. Selain itu juga pernah dilakukan penelitian oleh Adiria dengan format file *.BMP menggunakan bahasa pemrograman Delphi \cite{adiria}, sedangkan yang akan penulis buat nantinya adalah dengan mengkombinasikan kedua penelitian tersebut.

Penulis juga mencari jurnal-jurnal acuan dalam penelitian ini, seperti: \emph{Comparison of LSB Steganography in BMP and JPEG Images}; \emph{JPEG versus GIF Images in forms of LSB Steganography}; \emph{A Survey on Digital Image Steganography Techniques}; \emph{ A New LSB-S Image Steganography Method Blend with Cryptography for Secret Communication}, dan \emph{Steganography Using Least Signicant Bit Algorithm}. 

Dengan penjabaran di atas, penulis mengkombinasikan jurnal-jurnal tersebut untuk melakukan penelitian tentang "\textbf{Implementasi Steganografi pada Citral Digital dengan Metode \emph{Least Significant Bit}}". Pada penelitian ini format \emph{file} citra \emph{digital} yang dapat digunakan untuk menyimpan pesan adalah berformat *.bmp, dan citra hasil keluarannya juga berformat *.bmp. Pesan yang dapat disimpan adalah pesan teks. Dengan adanya penelitian ini diharapkan dapat memberikan informasi mengenai steganografi.

\section{Perumusan Masalah}
Rumusan masalah berdasarkan latar belakang di atas adalah:
\begin{enumerate}
	\item Bagaimana cara mengimplementasikan steganografi dengan metode \emph{Least Significant Bit} ke dalam citra \emph{digital}?
	\item Bagaimana perubahan dalam \emph{file} citra hasil keluaran sebelum dan sesudah disisipkan pesan teks?
\end{enumerate}

\section{Tujuan Penelitian}
Tujuan dari penelitian ini adalah: 
\begin{enumerate}
	\item Memberikan informasi bagaimana steganografi dapat diimplementasikan ke dalam citra \emph{digital} dengan menggunakan metode \emph{Least Significant Bit}. 
	\item Mengetahui perubahan yang terjadi dari hasil keluaran \emph{file} citra \emph{digital}.
\end{enumerate}

\section{Manfaat Penelitian}
Penelitian ini diharapkan memberikan manfaat sebagai berikut:
	\begin{enumerate}
		\item Bagi Penulis, diharapkan dapat mengetahui cara mengimplementasikan steganografi pada citra \emph{digital}, dan dapat mengetahui perubahan yang terjadi pada \emph{file} citra.
		\item Bagi Program Studi Ilmu Komputer, penulisan penelitian ini dapat memberikan gambaran bagi seluruh mahasiswa khususnya bagi mahasiswa program studi Ilmu Komputer Universitas Negeri Jakarta tentang proses dari steganografi, bagaimana steganografi dapat mengamankan \emph{file} dengan aman.
		\item Bagi Masyarakat, diharapkan dapat menjadi salah satu solusi dalam mengamankan \emph{file} mereka dari orang-orang yang tidak mempunyai hak untuk melihatnya.   	
	\end{enumerate}
		
% Baris ini digunakan untuk membantu dalam melakukan sitasi
% Karena diapit dengan comment, maka baris ini akan diabaikan
% oleh compiler LaTeX.
\begin{comment}
\bibliography{daftar-pustaka}
\end{comment}
